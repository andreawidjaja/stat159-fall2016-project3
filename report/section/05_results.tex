\documentclass{article}

\usepackage{Sweave}
\begin{document}
\Sconcordance{concordance:05_results.tex:05_results.Rnw:%
1 2 1 1 0 15 1}


\section{Results}
<<<<<<< HEAD
Once we figured out which variable selection methods we wanted to use and how we wanted to use them, we were then able to implement them using a Shiny R App. Our Shiny R App requires the following information from the user in order to run:\\

1) Type of Variable Selection Method\\
2) Type of Graduation Rate\\
3) Number of Relevant Variables to Output\\

The Shiny R App will then output the specified number of relevant variables for the specified graduation rate. For instance, lets say we were interested in finding the 5 most relevant explanatory variables for the overall graduation rate. Then, we would get the following output if we were to use all 3 variable selection methods:\\
=======
Once we figured out which variable selection methods we wanted to use and how we wanted to use them, we were then able to implement them using a Shiny R App. Our Shiny R App requires the following information from the user in order to run:

1) Type of Variable Selection Method
2) Type of Graduation Rate
3) Number of Relevant Variables to Output

The Shiny R App will then output the specified number of relevant variables for the specified graduation rate. For instance, lets say we were interested in finding the 5 most relevant explanatory variables for the overall graduation rate. Then, we would get the following output if we were to use all 3 variable selection methods:

>>>>>>> 5a31118e9819944f44f2c4d3acdbfc2e1a96aeeb

